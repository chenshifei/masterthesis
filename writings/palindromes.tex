\documentclass[thesis,fonts=libertine]{cluu}

\usepackage[style=cluu]{biblatex}
\addbibresource{palindromes.bib}

\usepackage{pythonhighlight}    % for \inputpython at the end

\begin{document}
\author{Per Starbäck}
\supervisors{Prof. Balthazar, Uppsala University\\
  	     Some Other Supervisor, NLP Enterprises, Inc.}
\title{Palindromes}
\subtitle{Never odd or even}

\maketitle

\begin{abstract}
  The concept of \emph{palindromes} is introduced, and some method for
  finding palindromes is developed.
\end{abstract}

\tableofcontents

\addchap{Preface}

I want to thank Donald Knuth for making \TeX, without which
I wouldn't have written this.

% by using \addchap instead of \chapter this preface isn't numbered.

\chapter{Introduction}

Palindromes are fun. I've tried to find some.
In Chapter \ref{chap:prev} previous work is reviewed, and
Chapter \ref{chap:results} is about my results.

\chapter{Previous work}
\label{chap:prev}

The longest palindromic word in the \emph{Oxford English Dictionary}
is the onomatopoeic \emph{tattarrattat}, coined by
\textcite{joyce:ulysses} for a knock on the door.
There is a growing literature where lots of palindromes are collected
\parencite{chism92:a_to_z, bergerson73},
and \textcite{funny} lists some surprisingly funny ones.

% From https://en.wikipedia.org/wiki/Palindrome
In computation theory the \emph{palindromic density} of an infinite
word \( w \) over an alphabet \( A \) is defined to be zero if only
finitely many prefixes are palindromes; otherwise, letting the
palindromic prefixes be of lengths \( n_k \) for \( k=1, 2, \dots\) we
define the density to be 

\begin{equation}
  d_P(w) =
  \left(\limsup\limits_{k\rightarrow\infty}\frac{n_{k+1}}{n_k} \right)^{-1}
\end{equation}

Among aperiodic words, the largest possible palindromic density is
achieved by the Fibonacci word, which has density \( 1/\varphi \), where
\( \varphi \)  is the Golden ratio \parencite[443]{adamczewski10}.

\chapter{Results}
\label{chap:results}

I examined a list of first names, and found a few there:
\emph{Anna, Hannah} and \emph{Otto}.

I have also made a program that searches for anagrams. The full
program is listed in appendix \ref{chap:code} on page
\pageref{chap:code}.

\appendix
\chapter{Code}
\label{chap:code}

\inputpython{pal.py}{1}{500}

\printbibliography
\end{document}
